\documentclass[]{jlreq}

\usepackage{graphicx}
\usepackage{amsmath,amssymb,amsthm}
\usepackage{bm}

\renewcommand{\figurename}{Fig.~}
\renewcommand{\tablename}{Table~}

\begin{document}
\title{Standard Map}
\author{Shin-ichi YOSHIMOTO}
\maketitle
\tableofcontents
\clearpage

\section{Kicked rotator}
%
\begin{equation}
    \mathcal{H}(\theta, p, t) = \frac{1}{2}p^2 + K \cos\theta \sum_{n=-\infty}^{\infty}\delta(t - n T)
\end{equation}
%
\begin{align}
    \frac{dp}{dt} &= -\frac{\partial \mathcal{H}}{\partial \theta} = K \sin\theta \sum_{n=-\infty}^{\infty}\delta(t - n T)\\
    \frac{d\theta}{dt} &= \frac{\partial H}{\partial p} = p
\end{align}
%
\section{Standard Map}
%
\begin{equation}
    \left\{
    \begin{aligned}
        p_{n+1} &= p_n + K\sin\theta_n \\
        \theta_{n+1} &= \theta_n + p_{n+1}
        \label{standrd_map}
    \end{aligned}
    \right.
\end{equation}
%
演算子$\tilde{h}$を
%
\begin{equation}
    \tilde{h}\theta_n \equiv K \sin\theta_n
\end{equation}
%
と定義すると、式(\ref{standrd_map})は
%
\begin{equation}
    \begin{pmatrix}
        p_{n+1}\\
        \theta_{n+1}
    \end{pmatrix}
    =
    \begin{pmatrix}
        1 & \tilde{h} \\
        1 & 1+\tilde{h}
    \end{pmatrix}
    \begin{pmatrix}
        p_n \\
        \theta_n
    \end{pmatrix}
\end{equation}
%
と書ける。

変換$(p_n, \theta_n)\mapsto (p_{n+1}, \theta_{n+1})$に対するJacobianは、
%
\begin{thebibliography}{9}
    \bibitem{oho}
    過去のOHOセミナーの教科書
    \bibitem{Ichikawa}
    市川芳彦, プラズマにおける非線形現象の諸問題 (3), 核融合研究, 1988, 59巻, 5号, p. 362-391.
  \end{thebibliography}
%
\end{document}