\documentclass[]{jlreq}

\usepackage{graphicx}
\usepackage{amsmath,amssymb,amsthm}
\usepackage{siunitx}
\usepackage{physics}
\usepackage{bm}

\renewcommand{\today}{\the\year/\the\month/\the\day}
\renewcommand{\contentsname}{Contents}
\renewcommand{\refname}{References}
\renewcommand{\figurename}{Fig.~}
\renewcommand{\tablename}{Table~}

\begin{document}
\title{Discrete-time Dynamical System}
\author{Shin-ichi YOSHIMOTO}
\maketitle
\tableofcontents
\clearpage

\section{Kicked rotator}
%
\begin{equation}
    \mathcal{H}(\theta, p, t) = \frac{1}{2}p^2 + K \cos\theta \sum_{n=-\infty}^{\infty}\delta(t - n T)
\end{equation}
%
\begin{align}
    \frac{dp}{dt} &= -\frac{\partial \mathcal{H}}{\partial \theta} = K \sin\theta \sum_{n=-\infty}^{\infty}\delta(t - n T)\\
    \frac{d\theta}{dt} &= \frac{\partial H}{\partial p} = p
\end{align}
%
\section{Standard Map}
%
\begin{equation}
    \left\{
    \begin{aligned}
        p_{n+1} &= p_n + K\sin\theta_n \\
        \theta_{n+1} &= \theta_n + p_{n+1}
        \label{standrd_map}
    \end{aligned}
    \right.
\end{equation}
%
演算子$\tilde{h}$を
%
\begin{equation}
    \tilde{h}\theta_n \equiv K \sin\theta_n
\end{equation}
%
と定義すると、式(\ref{standrd_map})は
%
\begin{equation}
    \begin{pmatrix}
        p_{n+1}\\
        \theta_{n+1}
    \end{pmatrix}
    =
    \begin{pmatrix}
        1 & \tilde{h} \\
        1 & 1+\tilde{h}
    \end{pmatrix}
    \begin{pmatrix}
        p_n \\
        \theta_n
    \end{pmatrix}
\end{equation}
%
と書ける。

変換$(p_n, \theta_n)\mapsto (p_{n+1}, \theta_{n+1})$に対するJacobianは、

\section{Linearization of a map at a fixed point}
%
\begin{equation}
    \left\{
    \begin{aligned}
        x_{n+1} &= f(x_n, y_n) \\
        y_{n+1} &= g(x_n, y_n)
        \label{map1}
    \end{aligned}
    \right.
\end{equation}
%
固定点$(x^*,y^*)$に対して
%
\begin{equation}
    \left\{
    \begin{aligned}
        f(x^*, y^*) &= x^* \\
        g(x^*, y^*) &= y^*
        \label{fixedpoints}
    \end{aligned}
    \right.
\end{equation}
%
$f(x,y)$と$g(x,y)$を$(x^*,y^*)$の周りでテーラー展開すると
%
\begin{equation}
    \left\{
    \begin{aligned}
        f(x,y) &= f(x^*, y^*) +f_x(x^*,y^*)(x-x^*)+f_y(x^*,y^*)(y-y^*)+ ... \\
        g(x,y) &= g(x^*, y^*) +g_x(x^*,y^*)(x-x^*)+g_y(x^*,y^*)(y-y^*)+ ...
    \end{aligned}
    \right.
\end{equation}
%
$\delta x_n = x_n - x^*,\; \delta y_n = y_n - y^*$と置くと、
$x_{n+1} = \delta x_{n+1} + x^*,\; y_{n+1} = \delta y_{n+1}+y^*$だから
%
\begin{equation}
    \left\{
    \begin{aligned}
        \delta x_{n+1} + x^* &= f(x^*, y^*) +f_x(x^*,y^*)\delta x_n+f_y(x^*,y^*)\delta y_n \\
        \delta y_{n+1} + y^* &= g(x^*, y^*) +g_x(x^*,y^*)\delta x_n + g_y(x^*,y^*)\delta y_n
    \end{aligned}
    \right.
\end{equation}
%
式(\ref{fixedpoints})より、
%
\begin{equation}
    \begin{pmatrix}
        \delta x_{n+1}\\
        \delta y_{n+1}
    \end{pmatrix}
    =
    \begin{pmatrix}
        f_x(x^*,y^*) & f_y(x^*,y^*)\\
        g_x(x^*,y^*) & g_y(x^*,y^*)
    \end{pmatrix}
    \begin{pmatrix}
        \delta x_n \\
        \delta y_n
    \end{pmatrix}
\end{equation}

\section{離散時間力学系}
\subsection{線形化方程式}

次の離散時間系を考える。
%
\begin{equation}
    x_i(k+1) = f_i(x_1(k),x_2(k),\dots, x_n(k)), \quad (i = 1,2,\dots,n)
\end{equation}
%
これはベクトル表記を用いて
%
\begin{equation}
    \bm{x}(k+1) = \bm{f}(\bm{x}(k))
    \label{map}
\end{equation}
%
と表すことにする。いま、$\bm{f}(x)$の固定点を$\bm{x}_0$とすると、
%
\begin{equation}
    \bm{f}(\bm{x}_0) = \bm{x}_0
\end{equation}
%
固定点からの微小な変動$\delta \bm{x}(k)$を考えると
%
\begin{equation}
    \bm{x}(k) = \bm{x}_0 + \delta \bm{x}(k)
\end{equation}
%
これを式(\ref{map})に代入して
%
\begin{equation}
    \bm{x}(k+1) = \bm{x}_0 + \delta \bm{x}(k+1) = \bm{f}(\bm{x}_0+\delta \bm{x}(k))
\end{equation}
%
\begin{equation}
    \bm{f}(\bm{x}_0+\delta \bm{x}(k)) = \bm{f}(\bm{x_0})+\frac{\partial}{\partial\bm{x}}\bm{f}(\bm{x_0})\delta \bm{x}(k) + \dots
\end{equation}
%
\begin{align}
    \delta\bm{x}(k+1) &= \frac{\partial}{\partial\bm{x}}\bm{f}(\bm{x_0})\delta \bm{x}(k) \\
        &= D\bm{f}(\bm{x_0})\delta\bm{x}(k)
\end{align}
%
\subsubsection{例: 2次元離散時間系}
%
\begin{thebibliography}{9}
    \bibitem{Arnold}
    アーノルド, 古典力学の数学的方法
    \bibitem{Ichikawa}
    市川芳彦, プラズマにおける非線形現象の諸問題 (3), 核融合研究, 1988, 59巻, 5号, p. 362-391.
    \bibitem{Kawakami}
    川上博, 非線形現象入門 定性的接近法, 2005
    \bibitem{Ito_1}
    伊藤大輔, 非線形力学系における分岐理論の解析・応用 I, システム/制御/情報, Vol. 64, No. 2, pp. 70-75, 2020
  \end{thebibliography}
%
\end{document}